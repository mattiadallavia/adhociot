\documentclass[a4paper,12pt]{article}

\usepackage[utf8]{inputenc}
\usepackage[T1]{fontenc}
\usepackage{lmodern}
\usepackage[italian]{babel}
\usepackage{graphicx}
\usepackage{amsthm}

\theoremstyle{definition}
\newtheorem{definition}{Definizione}

\begin{document}

% Nel contesto dell'IoT stanno maturando molteplici tecnologie di comunicazione wireless

Per rispondere al crescente interesse nei confronti dell'IoT, una nuova categoria di tecnologie di comunicazione wireless sta emergendo. Reti di sensori o oggetti autonomi pongono nuovi vincoli e necessità rispetto ai classici approcci di connettività. Al contempo, alcuni aspetti ritenuti classicamnete cruciali nella valutazione di una rete, assumono un'importanza secondaria e lasciano quindi spazio a nuovi modelli di funzionamento.

% LPWAN

La caratteristica che accomuna la grande maggioranza delle soluzioni IoT è la limitata disponibilità di energia: questi dispositivi sono infatti normalmente alimentati a batteria. Le principali soluzioni di connettività orbitano quindi attorno a questa recessità, focalizzandosi sul limitare i tempi di attività di trasmissione e ricezione.

La topologia adottata è quella a stella: ciascun elemento della rete comunica esculivamente con un gateway centrale.

\section{Modello e definizioni}

Si è scelto un semplice modello che descriva la rete dal punto di vista topologico.

\begin{figure}[h]
\centering
\includegraphics{model.pdf}
\caption{Esempio rappresentativo del modello.}
\end{figure}

Esso si compone di:

\begin{enumerate}
\item Un nodo centrale detto \emph{sink} che costituisce l'unico destinatario finale di tutti i dati.
\item Un numero $n$ di nodi satellite (sensori), con portata $p$ comune a tutti.
\item Un disco di raggio $r$ centrato nel \emph{sink}, detto ambiente, dove verranno distribuiti i satelliti.
\end{enumerate}

Appare subito evidente che la portata $p$ e il raggio dell'ambiente $r$ influenzano in maniera interdipendente il modello. Risultano quindi naturali le seguenti definizioni:

\begin{definition}
Sia $r$ il raggio ambiente e $p$ la portata dei nodi. Il \emph{guadagno lineare} $g$ è il loro rapporto:
$$ g = \frac{r}{p} $$
\end{definition}

\begin{definition}
Sia $r$ il raggio ambiente e $p$ la portata dei nodi. Il \emph{guadagno superficiale} $G$ è il rapporto tra l'area coperta da un singolo nodo e l'area totale dell'ambiente:
$$ G = \frac{\pi r^2}{\pi p^2} = \left(\frac{r}{p}\right)^2 = g^2 $$
\end{definition}

Queste definizioni rispondono alla seguente domanda: Fissata la portata dei nodi, quanto è più grande l'ambiente in cui sono disposti? In pratica queste grandezze normalizzano le misure di lunghezza e superficie. 

\end{document}