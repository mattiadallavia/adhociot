\documentclass[a4paper,12pt]{article}

\usepackage[utf8]{inputenc}
\usepackage[T1]{fontenc}
\usepackage{lmodern}
\usepackage[italian]{babel}
\usepackage{float}
\usepackage{graphicx}
\usepackage{subcaption}
\usepackage{amsmath}
\usepackage{amsthm}
\usepackage{amssymb}
\usepackage{xfrac}

\setlength{\parskip}{0.5em}

\theoremstyle{definition}
\newtheorem{definition}{Definizione}
\newtheorem{proposition}{Proposizione}

\title{Un algoritmo di routing di consegna ad-hoc nel cotesto dell'IoT}
\date{2018\\ Dicembre}
\author{Mattia Dalla Via\\ Dipartimento di Ignegneria dell'Informazione\\ Università degli Studi di Padova}

\begin{document}

\maketitle

\tableofcontents

\section{Premessa}

% Nel contesto dell'IoT stanno maturando molteplici tecnologie di comunicazione wireless

Per rispondere al crescente interesse nei confronti dell'IoT, una nuova categoria di tecnologie di comunicazione wireless sta emergendo. Reti di sensori o oggetti autonomi pongono nuovi vincoli e necessità rispetto ai classici approcci di connettività. Al contempo, alcuni aspetti ritenuti classicamnete cruciali nella valutazione di una rete, assumono un'importanza secondaria e lasciano quindi spazio a nuovi modelli di funzionamento.

% WSN - Wireless Sensor Network
% MANET - Mobile Ad-hoc Network
% LPWAN - Low Power Wide Area Network

La caratteristica che accomuna la grande maggioranza delle soluzioni IoT è la limitata disponibilità di energia: questi dispositivi sono infatti normalmente alimentati a batteria. Le principali soluzioni di connettività orbitano quindi attorno a questa recessità, focalizzandosi sul limitare i tempi di attività di trasmissione e ricezione.

La topologia adottata è quella a stella: ciascun elemento della rete comunica esculivamente con un gateway centrale.

\section{Introduzione}

\section{Modello della rete}

Cosa si intende esattamente quando si fa riferimento ad una \emph{rete di sensori}? In questa sezione viene descritto un modello che ne schematizza le caratteristiche fondamentali.

\subsection{Modello topologico}

Dal punto di vista spaziale, la rete è composta dai seguenti oggetti:

\begin{enumerate}
\item Un numero $n$ di nodi satellite (i sensori), ciascuno identificato da un indice $i \in \{1, \dots, n\}$ univoco.
\item Un nodo centrale detto \emph{sink}, di indice $i = 0$.
\item Un disco (regione di spazio) di raggio $r$, detto \emph{ambiente}, dove sono distribuiti i satelliti.
\end{enumerate}

Tutti i nodi hanno una portata (del segnale radio) $p$. Il nodo \emph{sink}, posizionato al centro dell'ambiente, costituisce il destinatario finale di tutti i messaggi. Questo nodo svolge il ruolo di \emph{gateway} verso il sistema informativo sovrastante.

\begin{figure}[H]
\centering
\includegraphics{model.pdf}
\caption{Rete di 3 satelliti, di cui 2 connessi al \emph{sink}.}
\end{figure}

\subsection{Grafo della rete}

Spesso si farà riferimento al \emph{grafo della rete}, definito come segue:

\begin{definition}
Siano $V = \{0, \dots, n \}$ gli indici dei nodi (\emph{sink} + $n$ satelliti). Siano $E$ gli archi (connessioni) tra i nodi, calcolati come:
$$ E = \{ (i, j, w) \ \text{dove} \ i, j \in V \ \text{e} \ w = dist(i, j) \mid dist(i, j) \le r \ \wedge \ i \neq j\} $$
Chiamaremo $R = (V, E)$ il \emph{grafo della rete}.
\end{definition}

La funzione $dist(i, j)$ rappresenta la distanza tra la coppia di nodi. Nelle successive simulazioni sarà definita come la distanza geometrica tra le coordinate. Nel caso queste non siano note, la distanza potrebbe essere stimata grazie alla potenza di ricezione dei messaggi.

Chiamaremo $C \subseteq R$ il sottografo contenente il \emph{sink} e diremo \emph{connessi al sink} tutti i nodo in esso contenuti, ovvero tutti quelli che hanno almeno un cammino che li collega al \emph{sink}. Se $C = R$ diremo che la rete è \emph{completamente connessa}.

\subsection{Relazioni caratteristiche}

Appare subito evidente che la portata e il raggio dell'ambiente influenzano in maniera interdipendente il modello. In particolare, fissati $r_0$ e $p_0$, è possibile definire una classe di reti dove $r = m \cdot r_0, \ p = m \cdot p_0$ con $m \in \mathbb{Q}_{>0}$ e dove le posizioni relative dei nodi sono conservate. Queste reti sono equivalenti a meno di un fattore di scala. È utile quindi definire il concetto di \emph{guadagno}, per identificare tutte quelle reti appartenenti a classi distinte.

\begin{definition}
Sia $r$ il raggio ambiente e $p$ la portata dei nodi. Il \emph{guadagno lineare} $g$ è il loro rapporto:
$$ g = \frac{r}{p} $$
\end{definition}

\begin{definition}
Sia $r$ il raggio ambiente e $p$ la portata dei nodi. Il \emph{guadagno superficiale} $G$ è il rapporto tra l'area coperta da un singolo nodo e l'area totale dell'ambiente:
$$ G = \frac{\pi r^2}{\pi p^2} = \left(\frac{r}{p}\right)^2 = g^2 $$
dove $g$ è il \emph{guadagno lineare}.
\end{definition}

Queste definizioni rispondono alla seguente domanda: Fissata la portata dei nodi, quanto è più grande l'ambiente in cui sono disposti? In pratica queste grandezze normalizzano le misure di lunghezza e superficie, usando $p$ come unità di misura.

Definiamo infine una speciale relazione tra $n$ e $g$ che sarà di centrale importanza per l'analisi delle proprietà di $R$:

\begin{definition}
Sia $n$ il numero di satelliti e $g$ il \emph{guadagno lineare}, chiamaiamo \emph{densità di guadagno superficiale} $d$ la relazione:
$$ d = \frac{n}{g^2} = \frac{n}{G} $$
Dove $G$ è il \emph{guadagno superficiale}.
\end{definition}

Questo rapporto ci da un'informazione sulla quantità di nodi per unità di superificie.

\subsection{Modello temporale}

Si è fatta l'ipotesi che tutti i nodi siano dotati di un orologio sincronizzato e siano perciò in grado di operare in modo coordinato.

Sono stati individuati i seguenti istanti:

\begin{enumerate}
\item $t_0 =$ tempo di accensione coordinata di tutti i nodi;
\item $t_f =$ istante dell'ultima trasmissione.
\end{enumerate}

\begin{figure}[H]
\centering
\includegraphics{timeline.pdf}
\caption{}
\end{figure}

\section{Analisi}

Muniti di una definizione formale di rete, carattarizzata dal grafo $R$ legato alla coppia $(n, g)$, andiamo ora ad individuare dei criteri qualitativi con i quali valutarla.

\subsection{Metriche}

Per fare ciò si ricorrerà all'uso delle seguenti definizioni:

\begin{definition}
Sia $D$ l'insieme dei possibili \emph{grafi di rete}, ciascuno associato ad una coppia $(n, g)$ dove $n \in \mathbb{N}$ e $r \in \mathbb{Q}_{>0}$. Chiameremo \emph{rapporto di connessione} la funzione
$$ c \colon D \to [0, 1] $$
\end{definition}

Questa risponde alla domanda: Dato un grafo $R \in D$, quanti dei satelliti della rete risultano connessi al nodo \emph{sink}? Ci da quindi la quantità (percentuale) dei nodi nel sottografo $C$ dei \emph{connessi al sink}.

Quale sarà il \emph{rapporto di connessione} al variare dei parametri $n$ e $p$? Questa domanda non trova una risposta univoca, in quanto esso fondamentalmente dipende dallo specifico \emph{grafo di rete} $R$.

Banalmente, se fissiamo $g = 1$, la porata del \emph{sink} coprirà l'intero ambiente e quindi tutti i satelliti saranno ad esso direttamente connessi, ovvero $c(R) = 1 \ \forall R$ con $g = 1$.

Posto invece $g > 1$, tutti i nodi potrebbero cadere nel disco di portata del \emph{sink} e quindi risultare $c(R_1) = 1$, oppure viceversa potrebboro essere tutti a distanza $> p$ dal \emph{sink} e perciò risultare $c(R_2) = 0$.

Nelle prossime sezioni approcceremo il problema con due distite strategie, tentando di fornire una risposta alla suddetta domanda.

\subsection{Rete a reticolo uniforme}

Ipotizzando che i nodi siano disposti in un reticolo uniforme, è possibile determinare la soglia per una rete completamente connessa.

Chiameremo \emph{rete a reticolo uniforme} $R_r$ il grafo di rete con queste caratteristiche.

\begin{figure}[H]
\centering
\includegraphics{model_unif.pdf}
\caption{}
\end{figure}

\begin{proposition}
Sia $R_r$ un \emph{grafo di rete} associato alla coppia $(n, g)$, dove i nodi sono disposti in un reticolo uniforme. Allora
$$ c(R_r) = 1 \ \text{se e solo se} \ n \geq \frac{4}{\pi} \left(\frac{2r}{p}\right)^2 = \frac{16}{\pi} G $$
Dove $G = g^2$ è il \emph{guadagno superficiale}.
$$ \text{In altre parole} \ d \geq \frac{16}{\pi} \simeq 5.09 $$
\end{proposition}

Questo risultato si ottiene facilmente considerando la porzione di area del cerchio di raggio $g = \sfrac{r}{p}$ inscritto in un quadrato di lato $2g = \sfrac{2r}{p}$. La proporzione tra l'area del quadrato e del cerchio inscritto è costante a $\sfrac{4}{\pi}$ mentre l'area del quadrato vale $(2g)^2 = 4G$. La disuguaglianza assicura infine che la distanza tra ogni coppia di nodi in asse sia $\leq p$.

Si ha quindi che in una rete $R_r$, per garantire la totale connessione, le coppie $(n, g)$ vanno scelte secondo la relazione $n \geq \sfrac{16}{\pi} g^2$.

\begin{figure}[H]
\centering
\includegraphics{ret_locus.pdf}
\caption{Luogo delle coppie $(n, g)$ per cui $c(R_r) = 1$}
\end{figure}

\subsection{Simulazione di reti campione}

L'idea è quella di produrre una sequenza di reti campione $R_i$, dove le coordinate dei nodi sono generate casualmente con probabilità uniforme. Calcolando il valor medio di $c(R_i)$ per ogni coppia di $(n, g)$, si è derivata una stima $\tilde{c}(n, g)$.

Da notare che questa stima non fornirà un informazione su quali probabilmente saranno i valori di $\tilde{c}(n, g)$, ma bensì su quali valori medi ci aspettiamo da un grande numero di reti eterogenee.

Per questa simulazione sono stati generati 100 grafi $R_i$ per ogni valore di $n \in \{0, 50, 100, 150, \dots, 1000\}$ e $g \in \{0, 1, 2, \dots, 20\}$ per un totale di $100 \cdot 20 \cdot 20 = 40000$ campioni.

\begin{figure}[H]
\begin{subfigure}[b]{0.5\textwidth}
\includegraphics[width=\textwidth]{conn_sim.pdf}
\caption{}
\end{subfigure}
\begin{subfigure}[b]{0.5\textwidth}
\includegraphics[width=\textwidth]{conn_g.pdf}
\caption{}
\end{subfigure}
\caption{Risultato della simulazione.}%
\end{figure}

Come è ragionevole immaginare, al crescere di $g$ il \emph{rapporto di connessione} cala repentinamente, nello specifico è inversamente proporzionale al \emph{guadagno superficiale} $G = g^2$. Questa tendenza è tuttavia mitigata al crescere di $n$, infatti all'aumentare dei satelliti cresce la probabilità che si formino dei percorsi al \emph{sink}.

Interpolando il dataset sulla funzione $\tilde{c}(d) = s \cdot tanh(w \cdot d + t) + q$ (determinata empiricamente) si ottiene:

$$ \tilde{c}(n, g) = 0.49 \cdot tanh(d - 4.53) + 0.51 $$

Dove $d$ è la \emph{densità di guadagno superficiale} $d = \frac{n}{G}$.

\begin{figure}[H]
\centering
\includegraphics{conn_den.pdf}
\caption{}
\end{figure}

Da questo grafico si puo vedere come un \emph{rapporto di connessione} medio $\tilde{c} \geq 0.9$ si può ottenere con $d \gtrsim 5.6$. Un risultato del tutto in linea con quello ottenuto per le \emph{reti a reticolo uniforme}.

\section{L'algoritmo}

\section{Prestazioni}

\section{Scenario applicativo}

\end{document}