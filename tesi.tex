\documentclass[a4paper,12pt]{article}

\usepackage[utf8]{inputenc}
\usepackage[T1]{fontenc}
\usepackage{lmodern}
\usepackage[italian]{babel}
\usepackage{graphicx}
\usepackage{amsthm}
\usepackage{amsfonts}

\setlength{\parskip}{1em}

\theoremstyle{definition}
\newtheorem{definition}{Definizione}

\title{Progettazione e analisi di un algoritmo di routing per una rete ad-hoc di consegna nel cotesto dell'IoT}
\date{2018\\ Dicembre}
\author{Mattia Dalla Via\\ Dipartimento di Ignegneria dell'Informazione\\ Università degli Studi di Padova}

\begin{document}

\maketitle

\tableofcontents

\section{Premessa}

% Nel contesto dell'IoT stanno maturando molteplici tecnologie di comunicazione wireless

Per rispondere al crescente interesse nei confronti dell'IoT, una nuova categoria di tecnologie di comunicazione wireless sta emergendo. Reti di sensori o oggetti autonomi pongono nuovi vincoli e necessità rispetto ai classici approcci di connettività. Al contempo, alcuni aspetti ritenuti classicamnete cruciali nella valutazione di una rete, assumono un'importanza secondaria e lasciano quindi spazio a nuovi modelli di funzionamento.

% LPWAN

La caratteristica che accomuna la grande maggioranza delle soluzioni IoT è la limitata disponibilità di energia: questi dispositivi sono infatti normalmente alimentati a batteria. Le principali soluzioni di connettività orbitano quindi attorno a questa recessità, focalizzandosi sul limitare i tempi di attività di trasmissione e ricezione.

La topologia adottata è quella a stella: ciascun elemento della rete comunica esculivamente con un gateway centrale.

\section{Modello della rete}

In questa sezione viene proposto un semplice modello che riassume le caratteristiche della rete. Sono individuati dei parametri caratteristici che serviranno a classificare, e in seguito analizzare, le possbili configurazioni spaziali e temporali dei nodi.

\subsection{Modello topologico}

Si focalizza sugli aspetti geometrici della rete, racchiudendo le informazioni di posizione portata e radio dei nodi.

\begin{figure}[h]
\centering
\includegraphics{model.pdf}
\caption{Schema di una possibile rete.}
\end{figure}

Esso si compone di:

\begin{enumerate}
\item Un nodo centrale detto \emph{sink} che costituisce l'unico destinatario finale di tutti i dati.
\item Un numero $n$ di nodi satellite (sensori), con portata $p$ comune a tutti.
\item Un disco di raggio $r$ centrato nel \emph{sink}, detto ambiente, dove verranno distribuiti i satelliti.
\end{enumerate}

Appare subito evidente che la portata $p$ e il raggio dell'ambiente $r$ influenzano in maniera interdipendente il modello. Risultano quindi naturali le seguenti definizioni:

\begin{definition}
Sia $r$ il raggio ambiente e $p$ la portata dei nodi. Il \emph{guadagno lineare} $g$ è il loro rapporto:
$$ g = \frac{r}{p} $$
\end{definition}

\begin{definition}
Sia $r$ il raggio ambiente e $p$ la portata dei nodi. Il \emph{guadagno superficiale} $G$ è il rapporto tra l'area coperta da un singolo nodo e l'area totale dell'ambiente:
$$ G = \frac{\pi r^2}{\pi p^2} = \left(\frac{r}{p}\right)^2 = g^2 $$
dove $g$ è il \emph{guadagno lineare}.
\end{definition}

Queste definizioni rispondono alla seguente domanda: Fissata la portata dei nodi, quanto è più grande l'ambiente in cui sono disposti? In pratica queste grandezze normalizzano le misure di lunghezza e superficie. 

La coppia numero di nodi, guadagno $(n, g)$ verrà d'ora in poi utilizzata come caratteristoica fondamentale di una rete.

\subsection{Modello temporale}

Si è fatta l'ipotesi che tutti i nodi siano dotati di un orologio sincronizzato e siano perciò in grado di operare in modo coordinato.

Sono stati individuati i seguenti istanti:

\begin{enumerate}
\item $t_0 =$ tempo di accensione coordinata di tutti i nodi;
\item $t_f =$ istante dell'ultima trasmissione.
\end{enumerate}

\section{Analisi}

Definiti i parametri caratteristici del modello, è stato importante comprendere come questi incidano sulla rete.

Per fare ciò si è ricorsi all'uso delle seguenti funzioni:

\begin{definition}
Sia $N = \{1, \dots, m\}$ un intervallo di numero di nodi con $m \in \mathbb{N}$ il numero massimo di nodi. Sia $D = \{x \in \mathbb{Q} \mid 0 \leq x \leq d\}$ un intervallo di guadagni con $d \in \mathbb{Q}$ il guadagno massimo. Chiameremo \emph{rapporto di connessione} la funzione
$$ c \colon N \times D \to [0, 1] $$
\end{definition}

Questa risponde alla domanda: Data una coppia di parametri $(n, g) \in N \times D$, quanti dei satelliti della rete risultano connessi al nodo \emph{sink}? Ci da quindi un'informazione sulla quantità (percentuale) dei nodi contenuti del sottografo contenente il \emph{sink}.

Considerando solamente i nodi connessi direttamente (da un solo salto) al \emph{sink}, questa funzione è definibile semplicemente considerando i nodi che cadono nel disco di portata del \emph{sink}, e risulta
\begin{equation}
c^{*}(n, g) = \frac{\pi p^2}{\pi r^2} = \frac{1}{g^2} = \frac{1}{G}
\end{equation}

Se $g = 1$ la porata del \emph{sink} coprirà l'intero ambiente e quindi tutti i satelliti saranno ad esso direttamente connessi, ovvero $c^{*}(n, 1) = 1 \ \forall g \in D$. 

\begin{definition}
Sia $N = \{1, \dots, m\}$ un intervallo di numero di nodi e sia $D = \{x \in \mathbb{Q} \mid 0 \leq x \leq d\}$ un intervallo di guadagni. Chiameremo \emph{probabilità di connessione} $p$ la funzione
$$ p \colon N \times D \to [0, 1] $$
\end{definition}

\begin{equation}
\tilde{c}(d) = 0.47 \cdot tanh(0.58 \cdot d - 2.29) + 0.53
\end{equation}

\end{document}