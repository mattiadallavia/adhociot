\documentclass[a4paper,12pt]{article}

\usepackage[utf8]{inputenc}
\usepackage[T1]{fontenc}
\usepackage{lmodern}
\usepackage[italian]{babel}
\usepackage{graphicx}
\usepackage{subcaption} 
\usepackage{amsmath}
\usepackage{amsthm}
\usepackage{amsfonts}

\setlength{\parskip}{0.5em}

\theoremstyle{definition}
\newtheorem{definition}{Definizione}

\title{Un algoritmo di routing di consegna ad-hoc nel cotesto dell'IoT}
\date{2018\\ Dicembre}
\author{Mattia Dalla Via\\ Dipartimento di Ignegneria dell'Informazione\\ Università degli Studi di Padova}

\begin{document}

\maketitle

\tableofcontents

\section{Premessa}

% Nel contesto dell'IoT stanno maturando molteplici tecnologie di comunicazione wireless

Per rispondere al crescente interesse nei confronti dell'IoT, una nuova categoria di tecnologie di comunicazione wireless sta emergendo. Reti di sensori o oggetti autonomi pongono nuovi vincoli e necessità rispetto ai classici approcci di connettività. Al contempo, alcuni aspetti ritenuti classicamnete cruciali nella valutazione di una rete, assumono un'importanza secondaria e lasciano quindi spazio a nuovi modelli di funzionamento.

% LPWAN

La caratteristica che accomuna la grande maggioranza delle soluzioni IoT è la limitata disponibilità di energia: questi dispositivi sono infatti normalmente alimentati a batteria. Le principali soluzioni di connettività orbitano quindi attorno a questa recessità, focalizzandosi sul limitare i tempi di attività di trasmissione e ricezione.

La topologia adottata è quella a stella: ciascun elemento della rete comunica esculivamente con un gateway centrale.

\section{Modello della rete}

In questa sezione viene proposto un modello della rete. Sono individuate le caratteristiche spaziali e temporali attorno alle quali questa tesi si concentra. Infine viene fornita una definizione formale di rete basata su un grafo pesato.

\subsection{Modello topologico}

Si compone di:

\begin{enumerate}
\item Un nodo centrale detto \emph{sink} che costituisce l'unico destinatario finale di tutti i dati.
\item Un numero $n$ di nodi satellite, con portata $p$ comune.
\item Un disco (regione di spazio) di raggio $r$ centrato nel \emph{sink}, detto ambiente, dove sono distribuiti i satelliti.
\end{enumerate}

A ciascun satellite è assegnato un indice $i \in \{1, \dots, n\}$ univoco e ed è posizionato alle coordinate (a lui ignote) $(x, y)$ nell'ambiente. Al nodo \emph{sink} è riservato l'indice $i = 0$ ed è sempre posizionato al centro dell'ambiente.

\begin{figure}[h]
\centering
\includegraphics{model.pdf}
\caption{Schema di una possibile rete.}
\end{figure}

Appare subito evidente che la portata $p$ e il raggio dell'ambiente $r$ influenzano in maniera interdipendente il modello. Risultano quindi naturale ridurle mediante le seguenti definizioni:

\begin{definition}
Sia $r$ il raggio ambiente e $p$ la portata dei nodi. Il \emph{guadagno lineare} $g$ è il loro rapporto:
$$ g = \frac{r}{p} $$
\end{definition}

\begin{definition}
Sia $r$ il raggio ambiente e $p$ la portata dei nodi. Il \emph{guadagno superficiale} $G$ è il rapporto tra l'area coperta da un singolo nodo e l'area totale dell'ambiente:
$$ G = \frac{\pi r^2}{\pi p^2} = \left(\frac{r}{p}\right)^2 = g^2 $$
dove $g$ è il \emph{guadagno lineare}.
\end{definition}

Queste definizioni rispondono alla seguente domanda: Fissata la portata dei nodi, quanto è più grande l'ambiente in cui sono disposti? In pratica queste grandezze normalizzano le misure di lunghezza e superficie, prendendo $p$ come unità di misura.

Spesso in questa tesi si farà riferimento al \emph{grafo della rete}, definito come segue:

\begin{definition}
Siano $V = \{1, \dots, n \}$ gli indici degli $n$ nodi. Siano $E$ gli archi (connessioni) tra i nodi, calcolati come:
$$ E = \{ (i, j, w) \ \text{dove} \ i, j \in V \ \text{e} \ w = dist(i, j) \mid dist(i, j) \le r \ \wedge \ i \neq j\} $$
dove $dist(i, j)$ è la distanza geometrica delle coordinate.\\
Chiamaremo $G = (V, E)$ il \emph{grafo della rete}.
\end{definition}

Il valore $w$ (peso dell'arco) fa riferimento alla distanza tra la coppia di nodi.
In questa definizione (come nelle successive simulazioni) si ricorre all'uso delle coordinate per il suo calcolo. In realtà non si è fatta l'ipotesi che la posizione dei nodi sia nota, in questo caso distanza andrà stimata grazie alla potenza di ricezione dei messaggi.

Questo oggetto rappresenta l'immagine di una rete e costituisce l'unico input per l'analisi e la simulazione della stessa.

\subsection{Modello temporale}

Si è fatta l'ipotesi che tutti i nodi siano dotati di un orologio sincronizzato e siano perciò in grado di operare in modo coordinato.

Sono stati individuati i seguenti istanti:

\begin{enumerate}
\item $t_0 =$ tempo di accensione coordinata di tutti i nodi;
\item $t_f =$ istante dell'ultima trasmissione.
\end{enumerate}

\section{Analisi}

Muniti di una definizione formale di rete, carattarizzata dal grafo $G$ legato alla coppia $(n, g)$, andiamo ora ad individuare dei criteri qualitativi con i quali valutarla.

\subsection{Metriche}

Per fare ciò si ricorrerà all'uso delle seguenti definizioni:

\begin{definition}
Sia $D$ l'insieme dei possibili \emph{grafi di rete}, ciascuno associato ad una coppia $(n, g)$ dove $n \in \mathbb{N}$ e $r \in \mathbb{Q}_{>0}$. Chiameremo \emph{rapporto di connessione} la funzione
$$ c \colon D \to [0, 1] $$
\end{definition}

Questa risponde alla domanda: Dato un grafo $G \in D$, quanti dei satelliti della rete risultano connessi al nodo \emph{sink}? Ci da quindi la quantità (percentuale) dei nodi nel sottografo contenente il \emph{sink}.

Considerando solamente i nodi connessi direttamente (da un solo salto) al \emph{sink}, questa funzione è definibile semplicemente considerando i nodi che cadono nel disco di portata del \emph{sink}, e risulta
\begin{equation}
c^{*}(n, g) = \frac{\pi p^2}{\pi r^2} = \frac{1}{g^2} = \frac{1}{G}
\end{equation}

Se $g = 1$ la porata del \emph{sink} coprirà l'intero ambiente e quindi tutti i satelliti saranno ad esso direttamente connessi, ovvero $c^{*}(n, 1) = 1 \ \forall g \in D$. 

\begin{definition}
Sia $N = \{1, \dots, m\}$ un intervallo di numero di nodi e sia $D = \{x \in \mathbb{Q} \mid 0 \leq x \leq d\}$ un intervallo di guadagni. Chiameremo \emph{probabilità di connessione} $p$ la funzione
$$ p \colon N \times D \to [0, 1] $$
\end{definition}

\subsection{Simulazione}

Quale sarà il \emph{rapporto di connessione} al variare dei parametri $n$ e $p$? Questa domanda non trova una risposta univoca, in quanto esso dipende dal \emph{grafo della rete}, ovvero dallo specifica disposizione dei nodi nell'ambiente.

Posto $g > 1$, tutti i nodi potrebbero cadere nel disco di portata del \emph{sink} e quindi risultare $c(G_1) = 1$, oppure viceversa potrebboro essere tutti a distanza $> p$ dal \emph{sink} e perciò risultare $c(G_2) = 0$.

Il valore di $c(n, g)$ risulta perciò impredicibile e fondamentalmente vincolato ad uno specifico grafo $G$.

La strategia utilizzata in questa tesi sarà quindi generare una sequenza di reti $G_i$, dove le coordinate dei nodi sono generate casualmente con probabilità uniforme. Calcolando il valor medio di $c(G_i)$ per la sequenza verrà prodotta una stima $\tilde{c}(n, r)$.

\begin{figure}[h]
\begin{subfigure}[b]{0.5\textwidth}
\includegraphics[width=\textwidth]{conn_sim.pdf}
\caption{}
\end{subfigure}
\begin{subfigure}[b]{0.5\textwidth}
\includegraphics[width=\textwidth]{conn_g.pdf}
\caption{}
\end{subfigure}
\caption{Risultato della simulazione.}%
\end{figure}

\begin{equation}
\tilde{c}(d) = 0.47 \cdot tanh(0.58 \cdot d - 2.29) + 0.53
\end{equation}

\end{document}