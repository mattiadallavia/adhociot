\documentclass[a4paper,12pt]{article}

\usepackage[utf8]{inputenc}
\usepackage[T1]{fontenc}
\usepackage{lmodern}
\usepackage[italian]{babel}
\usepackage{float}
\usepackage{graphicx}
\usepackage{subcaption}
\usepackage{amsmath}
\usepackage{amsthm}
\usepackage{amssymb}
\usepackage{xfrac}

\setlength{\parskip}{0.5em}

\theoremstyle{definition}
\newtheorem{definition}{Definizione}
\newtheorem{proposition}{Proposizione}

\title{Un paradigma per la raccolta dati in reti ad-hoc di sensori}
\date{2018\\ Dicembre}
\author{Mattia Dalla Via\\ Dipartimento di Ignegneria dell'Informazione\\ Università degli Studi di Padova}

% un paradigma ad-hoc per la consegna di dati in una rete di sensori
% radio, wireless, senza infrastruttura, efficiente, modello, ideazione e analisi, concezione e studio

\begin{document}

\maketitle

\clearpage

\tableofcontents

\section{Premessa}

% Nel contesto dell'IoT stanno maturando molteplici tecnologie di comunicazione wireless

Per rispondere al crescente interesse nei confronti dell'IoT, una nuova categoria di tecnologie di comunicazione wireless sta emergendo. Reti di sensori o oggetti autonomi pongono nuovi vincoli e necessità rispetto ai classici approcci di connettività. Al contempo, alcuni aspetti ritenuti classicamnete cruciali nella valutazione di una rete, assumono un'importanza secondaria e lasciano quindi spazio a nuovi modelli di funzionamento.

% WSN - Wireless Sensor Network
% MANET - Mobile Ad-hoc Network
% LPWAN - Low Power Wide Area Network

La caratteristica che accomuna la grande maggioranza delle soluzioni IoT è la limitata disponibilità di energia: questi dispositivi sono infatti normalmente alimentati a batteria. Le principali soluzioni di connettività orbitano quindi attorno a questa recessità, focalizzandosi sul limitare i tempi di attività di trasmissione e ricezione.

La topologia adottata è quella a stella: ciascun elemento della rete comunica esculivamente con un gateway centrale.

\section{Introduzione}

\section{Scenario applicativo}
% operatore mobile
% sensori sul territorio
% insfrastruttura su veicolo
% che copertura si ottiene

% finalità

\section{Modello}

Cosa si intende esattamente quando si fa riferimento ad una \emph{rete di sensori}? In questa sezione verrà descritto un modello che ne schematizza le caratteristiche fondamentali. Questo modello non ha lo scopo di rappresentare una rete in generale, bensì è calzato su misura per semplificare la descrizione dell'algoritmo.

\subsection{Nodi e ambiente}

% topologia
% punto di vista spaziale

Consideriamo una rete composta da $n+1$ nodi, ciascuno identificato da un indice $i \in \{0, \dots, n\}$ univoco. Il nodo di indice $i=0$, detto \emph{sink}, costituisce il destinatario finale di tutti i messaggi e svolge il ruolo di \emph{gateway} verso il sistema informativo sovrastante. Tutti gli altri nodi, chiamati \emph{satelliti}, sono i veri e propri sensori.

Chiamiamo \emph{portata} (del segnale) la distanza massima $p$ a cui è possibile trasmettere dati e supponiamo sia uguale per tutti i nodi. Inoltre, definiamo una regione di spazio \emph{ambiente} di raggio $a$ centrato nel \emph{sink}, nel quale sono distribuiti i satelliti.

\begin{figure}[H]
\centering
\includegraphics{figures/model.pdf}
\caption{Rete di 4 satelliti, di cui 3 raggiungibili dal \emph{sink}.}
\end{figure}

\subsection{Layout di rete}

Con \emph{layout di rete} ci riferiamo ad una specifica configurazione spaziale di $n$ satelliti di portata $p$ in un ambiente di raggio $a$. Parliamo quindi della ``fotografia'' di una rete che ritrae le coordinate di ciascun nodo, la loro portata e l'ampiezza dell'ambiente in cui sono disposti.

Ne diamo una definizione formale:

\begin{definition}
Sia
\begin{equation*}
N = \{(0,\,0,\,0),\ (1,\,x_1,\,y_1),\ \dots,\ (i,\,x_i,\,y_i),\ \dots,\ (n,\,x_n,\,y_n)\}
\end{equation*}
un insieme di $n+1$ nodi, di indice $i$ e coordinate $x_i,\,y_i$. Chiamiamo \emph{layout di rete} la tripla
\begin{equation*}
L = (a,\,p,\,N)
\end{equation*}
dove $a$ è il raggio dell'ambiente e $p$ la portata dei nodi.
\end{definition}

Da notare che il \emph{sink} ($i=0$) ha sempre coordinate $0,\,0$. Mentre per tutti i nodi vale che $dist(0,\,i) = \sqrt{x_i^2+y_i^2} \leq a$.

Vincolando i parametri $a$, $p$ e $n$ otteniamo un sottoinsieme di \emph{layout} dove solo le coordinate dei nodi rimangono indeterminate. Chiameremo questo sottoinsieme $layout(a,\,p,\,n)$. Nelle prossime sezioni esamineremo le relazioni che legano queste variabili e tenteremo di caratterizzare i sottoinsiemi che vanno a produrre. In altre parole, ci chiederemo: Fissati $a$, $p$ e $n$, cosa possiamo dire in generale sul \emph{layout}?

\subsection{Grafo di rete}

Noto un \emph{layout di rete} $L$, da esso si può estrapolare un grafo pesato. Il grafo descrive la connettività della rete, tenendo conto solamente delle distanze tra i nodi connessi tralasciando le loro posizioni.

\begin{figure}[H]
\centering
\includegraphics{figures/graph.pdf}
\caption{}
\end{figure}

Questo tipo di descrizione della rete si avvicina di più con quella a disposizione degli algoritmi di routing. Le coordinate dei nodi sono spesso infatti ignote, mentre le distanze sono stimabili.

Diamo anche per il grafo una definizione formale:

\begin{definition}
Siano $V = \{0,\,\dots,\,n\}$ gli indici di $n+1$ nodi con portata $p$. Siano $E$ gli archi (connessioni) tra i nodi, calcolati come:
\begin{equation*}
E = \{ (i,\,j,\,w) \text{ dove } i,\,j \in V \text{ e } w = \frac{dist(i,\,j)}{p} \mid w \le 1 \ \wedge \ i \neq j\}
\end{equation*}
Chiamiamo $G = (V,\,E)$ il \emph{grafo di rete}.
\end{definition}

%  associato al \emph{layout} $L$

La funzione $dist(i,\,j)$ rappresenta la distanza tra la coppia di nodi. Nelle successive simulazioni sarà definita come la distanza geometrica tra le coordinate.

% Nel caso queste non siano note, la distanza potrebbe essere valutata grazie alla potenza di ricezione dei messaggi.

Chiamaremo \emph{sottografo connesso} $C \subseteq G$ il sottografo contenente il \emph{sink} e diremo \emph{connessi al sink} tutti i nodo in esso contenuti, ovvero tutti quelli che hanno almeno un cammino che li collega al \emph{sink}. Se $C = R$ diremo che è \emph{completamente connesso}.

Diremo che $G = graph(L)$ è il grafo associato al \emph{layout} $L$ se questo è calcolato sulle coordinate dei nodi di $L$ e sulla loro portata.

Al contrario di quanto si potrebbe pensare, il passaggio per il concetto di \emph{layout} non risulta affatto superfluo. Basando la propria descizione direttamente su di un grafo si rischia infatti di descrivere situazioni che non rispettano i vincoli geometrici del piano. Ad esempio, noto che $dist(i,\,j) = dist(j,\,k) = 1$, allora necessariamente $0 \leq dist(i,\,k) \leq 2$. In altri termini potremmo dire che la funzione $graph(L)$ non è suriettiva. A luce di questo, chiameremo \emph{realistico} un grafo a cui è associato almeno un layout.

Dato un \emph{grafo realistico} $G$, esistono in realtà infiniti layout \emph{topologicamente equivalenti} tali che $G = graph(L_i)$ che vengono mappati sullo stesso grafo. In altre parole $graph(L)$ non è iniettiva. In particolare, dato
\begin{equation*}
L = (a,\,p,\,\{(0,\,0,\,0),\,(1,\,x_1,\,y_1),\,\dots,\,(i,\,x_i,\,y_i),\,\dots,\,(n,\,x_n,\,y_n)\})
\end{equation*}
chiamiamo \emph{layout in scala} $m$ il \emph{layout}
\begin{equation*}
scale(L,\,m) = (a\,\cdot\,m,\,p\,\cdot\,m,\,\{(0,\,0,\,0),\,\dots,\,(i,\,x_i\,\cdot\,m,\,y_i\,\cdot\,m),\,\dots\})
\end{equation*}
con $m \in \mathbb{Q}_{>0}$. I \emph{layout in scala} sono tutti \emph{topologicamente equivalenti} poiché
\begin{equation*}
w_m = \frac{dist(i_m,\,j_m)}{p_m} = \frac{dist(i,\,j)\,\cdot\,m}{p\,\cdot\,m} = \frac{dist(i,\,j)}{p} = w
\end{equation*}
ovvero i pesi degli archi si conservano. Consideriamo ad esempio
\begin{align*}
L_5 = (10,\,5,\,\{(0,\,0,\,0),\,(1,\,4,\,1),\,(2,\,-2,\,-3)\})\\
L_{15} = (30,\,15,\,\{(0,\,0,\,0),\,(1,\,12,\,3),\,(2,\,-6,\,-9)\})
\end{align*}
questi layout sono \emph{topologicamente equivalenti}, in particolare $L_{15} = scale(L_5,\,3)$. Entrambi vengono quindi mappati nello stesso grafo
\begin{equation*}
graph(L_5) = graph(L_{15}) = (\{0,\,1,\,2\},\,\{(0,\,1,\,0.82),\,(0,\,2,\,0.72)\})
\end{equation*}
Ci rendiamo quindi conto che sarà sufficiente studiare il sottoinsieme $layout(a,\,1,\,n)$ poiché qualsiasi layout $L_u$ con $p=u \neq 1$ sarà equivalente a $L_1 = scale(L_u,\,\frac{1}{u})$. Cioè $layout(a,\,p,\,n)$ è equivalente a $layout(\frac{a}{p},\,1,\,n)$.

\subsection{Dimensione e densità}

% relazioni caratteristiche

% Appare subito evidente che la portata e il raggio dell'ambiente influenzano in maniera interdipendente il modello. In particolare, fissati $r_0$ e $p_0$, è possibile definire una classe di reti dove $r = m \cdot r_0, \ p = m \cdot p_0$ con $m \in \mathbb{Q}_{>0}$ e dove le posizioni relative dei nodi sono conservate. Queste reti sono equivalenti a meno di un fattore di scala. È utile quindi definire il concetto di \emph{guadagno}, per identificare tutte quelle reti appartenenti a classi distinte.

% Formalizziamo ora questo concetto mediante le seguenti definizioni:

Alla luce di quanto appreso nella sezione precedente, introduciamo ora il concetto di \emph{dimensione del layout}. Questo esprime il raggio ambiente utilizzando la portata come unità di misura. In altre parole potremmo dire che è il raggio ambiente normalizzato su $p$.

\begin{definition}
Sia $a$ il raggio ambiente e $p$ la portata dei nodi. La \emph{dimensione radiale} $d$ è il loro rapporto:
\begin{equation*}
d = \frac{a}{p}
\end{equation*}
\end{definition}

\begin{definition}
Sia $a$ il raggio ambiente e $p$ la portata dei nodi. La \emph{dimensione superficiale} $D$ è il rapporto tra l'area coperta da un singolo nodo e l'area totale dell'ambiente:
\begin{equation*}
D = \frac{\pi a^2}{\pi p^2} = \left(\frac{a}{p}\right)^2 = d^2
\end{equation*}
\end{definition}

Nell'ottica dei \emph{layout in scala}, dato $L_p = (a,\,p,\,N)$, possiamo dire che $d$ è il raggio ambiente del layout equivalente $L_1 = (d,\,1,\,N)$. Come menzionato in precedenza sarà perciò sufficiente studiare i \emph{layout} al variare di $d$ ($layout(d,\,1,\,n)$) per coprire l'intero spazio dei \emph{grafi di rete}.

% Queste definizioni rispondono alla seguente domanda: Fissata la portata dei nodi, quanto è più grande l'ambiente in cui sono disposti? In pratica queste grandezze normalizzano le misure di lunghezza e superficie, usando $p$ come unità di misura.

Definiamo infine una speciale relazione tra $n$ e $d$ che sarà di centrale importanza per l'analisi delle proprietà di $L$:

\begin{definition}
Sia $n$ il numero di satelliti e $d$ la \emph{dimensione radiale}, chiamaiamo \emph{densità di nodi} $q$ la relazione:
\begin{equation*}
q = \frac{n}{d^2} = \frac{n}{D}
\end{equation*}
\end{definition}

Questo rapporto ci da un'informazione sulla quantità di nodi per unità di superificie.

\subsection{Tempo}

Si ipotizza che tutti i nodi siano dotati di un orologio sincronizzato e siano perciò in grado di accendersi contemporaneamente e operare in modo coordinato.

Per semplicità misureremo il tempo in modo che la durata di una trasmissione coincida con un quanto temporale. Una trasmissione iniziata a $t$ terminerà quindi entro $t+1$. Per il resto della tesi ci riferemo a $t=0$ come l'istante iniziale di accensione della rete e a $t$ come l'istante corrente.

Il tempo è suddiviso in \emph{frame} della durata di 3 quanti temporali. Chiameremo $frame(t) =  \left\lfloor\frac{t}{3}\right\rfloor$ il frame associato all'istante $t$. Ciascun \emph{frame} è suddiviso in 3 slot, indicizzati $0,\,1 \text{ e } 2$. Chiameremo $slot(t) = t \bmod 3$ lo slot associato all'istante $t$.

\begin{figure}[H]
\centering
\includegraphics{figures/timeline.pdf}
\caption{}
\end{figure}

Questa suddivisione sarà in seguito giustificata per il multiplexing temporale, parte del meccanismo di collision avoidance.

\subsection{Canale}

Il \emph{canale} è un numero naturale. Esso costituisce un'astrazione per il multiplexing di frequenza o un qualsiasi altro tipo di multiplexing che permette a due moduli di trasmettere contemporaneamente senza interferenze.

Ipotizziamo che all'istante $t$ stanno trasmettendo i due nodi $i$ e $j$. Se questi stanno trasmettendo sullo stesso canale ($channel(i) = channel(j)$) parleremo di \emph{collisione}, altrimenti entrambi i messaggi saranno consegnati con successo.

Da notare che, trattandosi di comunicazione wireless, due o più trasmissioni contemporanee sullo stesso canale non necessariamente collideranno in tutti i punti dello spazio.

\begin{figure}[H]
\centering
\includegraphics{figures/collision.pdf}
\caption{}
\end{figure}

In questo esempio, $i$ e $j$ stanno trasmettendo. $k$ si trova nella \emph{regione di collisione} e perciò non riceverà il messaggio. $q$ invece è raggiusto solamente dal segnale di $j$ e perciò non soffrirà dell'interferenza.

% modello radio

\section{Analisi}

Muniti di una definizione formale di rete, carattarizzata dal grafo $R$ legato alla coppia $(n, g)$, andiamo ora ad individuare dei criteri qualitativi con i quali valutarla.

\subsection{Metriche}

Per fare ciò si ricorrerà all'uso delle seguenti definizioni:

\begin{definition}
Sia $D$ l'insieme dei possibili \emph{grafi di rete}, ciascuno associato ad una coppia $(n, g)$ dove $n \in \mathbb{N}$ e $r \in \mathbb{Q}_{>0}$. Chiameremo \emph{rapporto di connessione} la funzione
\begin{equation*}
c \colon D \to [0, 1]
\end{equation*}
\end{definition}

Questa risponde alla domanda: Dato un grafo $R \in D$, quanti dei satelliti della rete risultano connessi al nodo \emph{sink}? Ci da quindi la quantità (percentuale) dei nodi nel sottografo $C$ dei \emph{connessi al sink}.

Quale sarà il \emph{rapporto di connessione} al variare dei parametri $n$ e $p$? Questa domanda non trova una risposta univoca, in quanto esso fondamentalmente dipende dallo specifico \emph{grafo di rete} $R$.

Banalmente, se fissiamo $g = 1$, la porata del \emph{sink} coprirà l'intero ambiente e quindi tutti i satelliti saranno ad esso direttamente connessi, ovvero $c(R) = 1 \ \forall R$ con $g = 1$.

Posto invece $g > 1$, tutti i nodi potrebbero cadere nel disco di portata del \emph{sink} e quindi risultare $c(R_1) = 1$, oppure viceversa potrebboro essere tutti a distanza $> p$ dal \emph{sink} e perciò risultare $c(R_2) = 0$.

Nelle prossime sezioni approcceremo il problema con due distite strategie, tentando di fornire una risposta alla suddetta domanda.

\subsection{Rete a dorsali}

Chiamiamo una \emph{rete a dorsali} un grafo dove i satelliti sono disposti in linea su $l$ raggi. I nodi sono disposti a catena e crescono all'aumentare di $r$, offrendo un considerevole vantaggio di copertura.

\begin{figure}[H]
\centering
\includegraphics{figures/model_backb.pdf}
\caption{}
\end{figure}

In una rete con $l$ direttive, è possibile coprire un ambiente con guadagno $g$ con $n = l \cdot g$ nodi.

Questa disposizione fornisce di gran lunga il maggior vantaggio in termini di copertura in profondità con una densità superficiale $d = \sfrac{n}{G} = \sfrac{l \cdot g}{g^2} = \sfrac{l}{g}$.

\subsection{Rete a reticolo uniforme}

Ipotizzando che i nodi siano disposti in un reticolo uniforme, è possibile determinare la soglia per una rete completamente connessa.

Chiameremo \emph{rete a reticolo uniforme} $R_r$ il grafo di rete con queste caratteristiche.

\begin{figure}[H]
\centering
\includegraphics{figures/model_unif.pdf}
\caption{}
\end{figure}

\begin{proposition}
Sia $R_r$ un \emph{grafo di rete} associato alla coppia $(n, g)$, dove i nodi sono disposti in un reticolo uniforme. Allora
\begin{equation*}
c(R_r) = 1 \ \text{se e solo se} \ n \geq \frac{4}{\pi} \left(\frac{2r}{p}\right)^2 = \frac{16}{\pi} G
\end{equation*}
Dove $G = g^2$ è il \emph{guadagno superficiale}.
\begin{equation*}
\text{In altre parole} \ d \geq \frac{16}{\pi} \simeq 5.09
\end{equation*}
\end{proposition}

Questo risultato si ottiene facilmente considerando la porzione di area del cerchio di raggio $g = \sfrac{r}{p}$ inscritto in un quadrato di lato $2g = \sfrac{2r}{p}$. La proporzione tra l'area del quadrato e del cerchio inscritto è costante a $\sfrac{4}{\pi}$ mentre l'area del quadrato vale $(2g)^2 = 4G$. La disuguaglianza assicura infine che la distanza tra ogni coppia di nodi in asse sia $\leq p$.

Si ha quindi che in una rete $R_r$, per garantire la totale connessione, le coppie $(n, g)$ vanno scelte secondo la relazione $n \geq \sfrac{16}{\pi} g^2$.

\begin{figure}[H]
\centering
\includegraphics{figures/ret_locus.pdf}
\caption{Luogo delle coppie $(n, g)$ per cui $c(R_r) = 1$}
\end{figure}

\subsection{Simulazione di reti campione}

L'idea è quella di produrre una sequenza di reti campione $R_i$, dove le coordinate dei nodi sono generate casualmente con probabilità uniforme. Calcolando il valor medio di $c(R_i)$ per ogni coppia di $(n, g)$, si è derivata una stima $\tilde{c}(d)$.

Da notare che questa stima non fornirà un informazione su quali probabilmente saranno i valori di $\tilde{c}(n, g)$, ma bensì su quali valori medi ci aspettiamo da un grande numero di reti eterogenee.

Per questa simulazione sono stati generati 100 grafi $R_i$ per ogni valore di $n \in \{0, 50, 100, 150, \dots, 1000\}$ e $g \in \{0, 1, 2, \dots, 20\}$ per un totale di $100 \cdot 20 \cdot 20 = 40\,000$ campioni.

\begin{figure}[H]
\begin{subfigure}[b]{0.5\textwidth}
\includegraphics[width=\textwidth]{figures/conn_sim.pdf}
\caption{}
\end{subfigure}
\begin{subfigure}[b]{0.5\textwidth}
\includegraphics[width=\textwidth]{figures/conn_g.pdf}
\caption{}
\end{subfigure}
\caption{Risultato della simulazione.}%
\end{figure}

Come è ragionevole immaginare, al crescere di $g$ il \emph{rapporto di connessione} cala repentinamente, nello specifico è inversamente proporzionale al \emph{guadagno superficiale} $G = g^2$. Questa tendenza è tuttavia mitigata al crescere di $n$, infatti all'aumentare dei satelliti cresce la probabilità che si formino dei percorsi al \emph{sink}.

Interpolando il dataset sulla funzione $\tilde{c}(d) = s \cdot tanh(w \cdot d + t) + q$ (determinata empiricamente) si ottiene:
\begin{equation*}
\tilde{c}(n, g) = 0.49 \cdot tanh(d - 4.53) + 0.51
\end{equation*}
Dove $d$ è la \emph{densità di guadagno superficiale} $d = \frac{n}{G}$.

\begin{figure}[H]
\centering
\includegraphics{figures/conn_den.pdf}
\caption{}
\end{figure}

Da questo grafico si puo vedere come un \emph{rapporto di connessione} medio $\tilde{c} \geq 0.9$ si può ottenere con $d \gtrsim 5.6$. Un risultato del tutto in linea con quello ottenuto per le \emph{reti a reticolo uniforme}.

\section{L'algoritmo}

L'algoritmo di routing è stato ideato attorno attorno ad alcuni specifici presupposti:

\begin{enumerate}
\item Infrastruttura ad-hoc
\item Minimizzare le trasmissioni
\item Configurare la rete in assenza di un coordinamento prestabilito
\item Affidabilità
\item Semplicità della routine di nodo
\end{enumerate}

\subsection{Principio di funzionamento}

L'algoritmo opera a macchia d'olio: A $t=0$ il \emph{sink} da inizio alle trasmissioni con un messaggio vuoto di discovery. Le trasmissioni a questo punto avvengono a partire dal centro verso la perifieria, procedento man mano in anelli concentrici. Durante questo processo il \emph{grafo di rete} viene esplorato, determinando implicitamente un albero che collega ciascun satellite al \emph{sink} ed assegnando ad ogni nodo la propria profondità (\emph{depth}).

Il vantaggio in efficienza è frutto della duplice valenza che assume ogni trasmissione: Consideriamo l'invio del messaggio $m$ da parte del nodo $i$ di profondità $p$, questa corrisponde all'inoltro del dato verso i nodi di profondità $p-1$, nella direzione del \emph{sink}. Inoltre, se si tratta della prima trasmissione del nodo $i$, essa funge da notifica agli eventuali nodi di profondità $p+1$, che verranno attivati. Viceversa, se il messaggio $m$ proviene da un nodo di profondità $> p$, si sta effettuando una ritrasmissione. In questo caso l'invio funge da riscontro di $m$ ai nodi di profondità $p+1$.

\begin{figure}[H]
\centering
\includegraphics{figures/net_example.pdf}
\caption{}
\end{figure}

\subsection{Variabili di stato}

Ciascun nodo fa uso dei seguenti campi:

\begin{table}[H]
\centering
\begin{tabular}{| l | l | l |}
\multicolumn{1}{l}{Nome} &
\multicolumn{1}{l}{Tipo} &
\multicolumn{1}{l}{Valore iniziale} \\ \hline
\texttt{index} & int & $i$ \\ \hline
\texttt{state} & int & $waiting$ \\ \hline
\texttt{depth} & int & $unknown$ \\ \hline
\texttt{channel} & int & $0$ \\ \hline
\texttt{attempts} & int & $0$ \\ \hline
\texttt{wait} & int & $frame(t)$ \\ \hline
\texttt{messages} & stack & $\{m\}$ \\ \hline
\end{tabular}
\caption{}
\end{table}

\texttt{index} è un identificativo univoco, assegnato a priori.

\texttt{state} controlla la modalità di funzionamento del nodo. Inizialemnte assume il valore di $waiting$: il satellite è in attesa di venire attivato da un messaggio discovery. Una volta attivato ($state(i) = active$), viene abilitata la routine di trasmissione.

\texttt{depth} indica la profondità di $i$, ovvero il numero di archi che lo collega al \emph{sink}. Viene determinata alla prima ricezione come:
\begin{equation*}
depth(i) = depth(j) + 1
\end{equation*}

\texttt{channel} seleziona il canale di trasmissione, è inizializzato a $0$ ed incrementato fino a che non viene consegnato con successo un messaggio.

\texttt{attempts} conteggia il numero di tentativi di trasmissione non ancora confermati. Serve a calcolare il frame di ritrasmissione mediante l'exponential backoff.

\texttt{wait} è il numero del frame al quale si è pianificata la prossima trasmissione.

\texttt{messages} è la pila dei messaggi da inviare (compresi quelli inviati ma non ancora confermati). A $t=0$ viene inizializzata con un messaggio $m$ con $number(m) = i$, questo garantisce l'unicità dell'identificatore del messaggio.

\subsection{Messaggio}

Ogni messaggio $m$, inviato dal nodo $i$ di profondità $p$, svolge implicitamente almeno uno di questi ruoli:

\begin{enumerate}
\item \emph{consegna} del payload ad un nodo di profondità $p-1$
\item \emph{esplorazione} dei nodi a profondità $p+1$, che, ricevendo $m$ si attivano.
\item \emph{riscontro} della ricezione di $m$ al nodi di origine di profondità $p+1$.
\end{enumerate}

\begin{figure}[H]
\centering
\includegraphics{figures/tx.pdf}
\caption{}
\end{figure}

Oltre ovviamente al \emph{payload}, sono inviate le seguenti informazioni:

\begin{table}[H]
\centering
\begin{tabular}{| l | l | l |}
\multicolumn{1}{l}{Nome} &
\multicolumn{1}{l}{Tipo} &
\multicolumn{1}{l}{Valore iniziale} \\ \hline
\texttt{number} & int & $n$ \\ \hline
\texttt{depth} & int & $d$ \\ \hline
\texttt{node\_confirm} & int & $null$ \\ \hline
\texttt{channel\_confirm} & int & $null$ \\ \hline
\end{tabular}
\caption{}
\end{table}

\texttt{number} è un identificatore univoco del messaggio. È inizializzato con l'indice del nodo di origine.

\texttt{depth} è la profondità del nodo che invia il messaggio.

\texttt{node\_confirm} è l'indice del satellite da cui il messaggio è stato ricevuto. Funge da acknowledge ai nodi a $p-1$.

\texttt{channel\_confirm} è il canale su cui il messaggio è stato ricevuto. Serve ai nodi a $p-1$ per spartirsi mutualmente i canali.

\subsection{Routine di ricezione}

In nodo $i$ riceve, all'istante $t_r$, il mesaggio $m$ dal nodo $j$.

Se $state(i) = waiting$, il nodo viene attivato ($state(i) = active$), abilitando quindi $i$ a trasmettere.

La profondità viene ricalcolata come
\begin{equation*}
depth(i) = min(depth(i),\ depth(j)+1)
\end{equation*}

Se $depth(j) > depth(i)$, il messaggio arriva da un nodo più periferico. Lo scheduliamo perciò per una ritrasmissione: $push(i,\ m)$, con l'accortezza di eliminare eventuali vecchie istanze dello stesso messaggio dalla stack.

\begin{figure}[H]
\centering
\includegraphics{figures/rx_farther.pdf}
\caption{}
\end{figure}

Se invece $depth(j) < depth(i)$ e $m \in messages(i)$, stiamo ricevendo un riscontro per un messaggio precedentemente inoltrato. $m$ è stato ricevuto con successo da un nodo più vicino al \emph{sink}, possiamo quindi procedere a eliminarlo dalla stack: $remove(i,\ m)$.

\begin{figure}[H]
\centering
\includegraphics{figures/rx_closer.pdf}
\caption{}
\end{figure}

\subsection{Schedulazione}

I satelliti appena attivati da un messaggio di discovery dal nodo $j$ corrono il rischio di effettuare la prima trasmissione sullo stesso frame e canale. Per mitigare questo fenomeno, i nodi appena attivati schedulano la propria trasmissione mediante un criterio euristico basato sulla distanza da $j$ (eventualmente stimata mediante la potenza ricevuta):
\begin{equation*}
wait(i) = frame(t) + dist(i,\ j)
\end{equation*}

Se $node\_confirm(m) = i$, stiamo ricevendo il riscontro di una precedente trasmissione di $m$. Va quindi azzerato il contatore delle trasmissioni senza successo $attempts(i) = 0$ e rischedulate le trasmissioni al prossimo frame disponibile $wait(i) = frame(t)$.

Inviato un messaggio, finché non si riceve un riscontro valido, si assume che la trasmissione non sia andata a buon fine. Perciò si incrementa il contatore dei tentativi di trasmissione $attempts(i)$\textit{++} e si rischedula una ritrasmissione mediante \emph{exponential backoff} al frame:
\begin{equation*}
wait(i) = frame(t) + rand(1,\ 2^a)
\end{equation*}
con $a = attempts(i)$ e $rand(a,\ b)$ un numero casuale tra $a$ e $b$.

\subsection{Contesa di canale}

Se $depth(j) < depth(i)$, $node\_confirm(m) \neq i$ e $channel\_confirm(m) = channel(i)$, significa che un altro nodo adiacente (di profondità $depth(i)$) ha consegnato con successo un messaggio sul canale $channel(i)$. È perciò necessario selezionare un nuovo canale: $channel(i)$\textit{++}.

\subsection{Routine di trasmissione}

% complessità
% efficienza
% ciclo di vita
% collisioni / contesa / exponential backoff
% gestione di canale
% schedulazione
% riscontro
% nozione di profindità
% relay
% stack dei messaggi
% ciascun messaggio può essere di discovery,riscontro,consegna
% formato del messaggio

\section{Simulazione}

% time loop

\subsection{Esempio 1}

\subsection{Prestazioni}

\end{document}